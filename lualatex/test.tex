\documentclass{article}

\usepackage{luatexja}
\usepackage{luatexja-fontspec}

\setmainfont{TeXGyreTermes}
\setsansfont{TeXGyreHeros}

\setmainjfont[BoldFont=SimHei]{SimSun}
\setsansjfont{SimHei}

\begin{document}
我第二次到仙岩的时候,我惊诧于梅雨潭的绿了。

梅雨潭是一个瀑布潭。仙瀑有三个瀑布,梅雨瀑最低。走到山边,便听见花花花花的声音;抬起头,镶在两条湿湿的黑边儿里的,一带白而发亮的水便呈现于眼前了。

我们先到梅雨亭。梅雨亭正对着那条瀑布;坐在亭边,不必仰头,便可见它的全体了。亭下深深的便是梅雨潭。这个亭踞在突出的一角的岩石上,上下都空空儿的;仿佛一只苍鹰展着翼翅浮在天宇中一般。三面都是山,像半个环儿拥着;人如在井底了。这是一个秋季的薄阴的天气。微微的云在我们顶上流着;岩面与草丛都从润湿中透出几分油油的绿意。而瀑布也似乎分外的响了。那瀑布从上面冲下,仿佛已被扯成大小的几绺;不复是一幅整齐而平滑的布。岩上有许多棱角;瀑流经过时,作急剧的撞击,便飞花碎玉般乱溅着了。那溅着的水花,晶莹而多芒;远望去,像一朵朵小小的白梅,微雨似的纷纷落着。据说,这就是梅雨潭之所以得名了。但我觉得像杨花,格外确切些。轻风起来时,点点随风飘散,那更是杨花了。——这时偶然有几点送入我们温暖的怀里,便倏的钻了进去,再也寻它不着。

梅雨潭闪闪的绿色招引着我们;我们开始追捉她那离合的神光了。揪着草,攀着乱石,小心探身下去,又鞠躬过了一个石穹门,便到了汪汪一碧的潭边了。瀑布在襟袖之间;但我的心中已没有瀑布了。我的心随潭水的绿而摇荡。那醉人的绿呀,仿佛一张极大极大的荷叶铺着,满是奇异的绿呀。我想张开两臂抱住她;但这是怎样一个妄想呀。--站在水边,望到那面,居然觉着有些远呢!这平铺着,厚积着的绿,着实可爱。她松松的皱缬着,像少妇拖着的裙幅;她轻轻的摆弄着,像跳动的初恋的处女的心;她滑滑的明亮着,像涂了“明油”一般,有鸡蛋清那样软,那样嫩,令人想着所曾触过的最嫩的皮肤;她又不杂些儿法滓,宛然一块温润的碧玉,只清清的一色--但你却看不透她!我曾见过北京什刹海指地的绿杨,脱不了鹅黄的底子,似乎太淡了。我又曾见过杭州虎跑寺旁高峻而深密的“绿壁”,重叠着无穷的碧草与绿叶的,那又似乎太浓了。其余呢,西湖的波太明了,秦淮河的又太暗了。可爱的,我将什么来比拟你呢?我怎么比拟得出呢?大约潭是很深的、故能蕴蓄着这样奇异的绿;仿佛蔚蓝的天融了一块在里面似的,这才这般的鲜润呀。——那醉人的绿呀!我若能裁你以为带,我将赠给那轻盈的舞女;她必能临风飘举了。我若能挹你以为眼,我将赠给那善歌的盲妹;她必明眸善睐了。我舍不得你;我怎舍得你呢?我用手拍着你,抚摩着你,如同一个十二三岁的小姑娘。我又掬你入口,便是吻着她了。我送你一个名字,我从此叫你“女儿绿”,好么?

我第二次到仙岩的时候,我不禁惊诧于梅雨潭的绿了。 
\end{document}