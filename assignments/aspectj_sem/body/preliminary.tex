\section{Preliminary}

\subsection{Terminologies}
There are some basic terminologies in program with AOP/aspectj:
\begin{description}
\item[jointpoint] well-defined points in the execution of a program.It requires that the joinpoints are identifiable by the AOP tools.Those join points include class constructions,get/set operations,method calls,method executions etc.
\item[Pointcut] a constructor that designates a set of join points defined in \emph{aspectj file}(file with suffix of .aj).
\item[advice] pieces of code attached to pointcuts and executed when join points satisfying their pointcuts are reached.Formally it is written as:$Advice=kind\times pointcut\times code$,where $kind\in\{before,after,around\}$.
\item[aspect] set of advices.
\end{description}

\subsection{Aspectj Weaving Mechinism}
Since \texttt{Aspectj1.1}(the current version is 1.5),the advice weaving has been based on bytecode transformation rather than on source code. In this case,the AspectJ compiler(ajc for short) is composed of two stages.
\begin{description}
\item[front-end compiler] an extension to the Java compiler and compiles applications and aspects into pure Java bytecode enriched with additional annotations to handle non pure Java information as advices and pointcuts.
\item[back-end compiler] weaves compiled aspects with compiled applications producing woven class files.
\end{description}

\subsection{Pointcut Designators}
As to compile time information access,there are two kinds of pointcuts:the \textsl{static} pointcut and the \textsl{dynamic} one.While static pointcut can be directly mapped to the original code, we have to include conditional logic(called \texttt{residuals}) to check the dynamic properties with the dynamic pointcuts.This paper only talks about the former.Refer to more details in \cite{masterAspectj} for more infomation.\\
So we only consider the designators below 8 kinds of pointcuts.
\begin{table*}[!hbtp]
  \centering
\begin{tabular}{|c|c|c|}
\hline \textbf{designator}&\textbf{matches}&\textbf{Name in this paper}\\
\hline call&calls to a \texttt{method} or constructor& \texttt{mcall}\\
\hline execution&execution of a method or constructor& \texttt{mexecution}\\
\hline get&the reference to a class attribute&\texttt{get}\\
\hline set&the assignment of a class attribute&\texttt{set}\\
\hline staticinitialization&execution of a class's static initialization&\texttt{staticInit}\\
\hline adviceexecution&on advice join points&\texttt{aexecution}\\
\hline withincode&join points within a method/constructor&\texttt{withincode}\\
\hline within& join points within a specific class type&\texttt{within}\\
\hline
\end{tabular}
\caption{static pointcut designators}
\end{table*}
The regular expression patterns for pointcuts such as \texttt{some\_method(..)} or \texttt{*} is also omitted to ease the semantics reading since they can be extended accurately in compile time.


\subsection{JVM Language Instructions}
To make it simpler,we also use a subset of JVMLI for our research.In this case,those instructions used for efficiency in the JVM specification is also ignored(such as \texttt{iconst\_0});also it omits some primitive type for JVM such as float,array and disable the cast operations(so there are no more \texttt{anewarray} or \texttt{d2l}).Also the parts of the other bytes are simplified as numbers($i$) or address($adr$) for short. The instructions we used is as below:
\begin{table}[hbt]
\centering
\begin{tabular}{cccc}
\hline\hline pop & push n & dup &iadd \\
\hline return & ireturn & areturen & athrow \\
\hline aload i & iload i & astore i&istore i \\
\hline getfield i& putfield i &getstatic i &  putstatic i\\
\hline invokeinterface i,n & invokestatic i & ifeq adr & ifne adr\\
\hline invokevirtual i & invokespecial i & goto adr  & new i \\
\hline monitorenter & monitorexit\\
\hline
\end{tabular}
\caption{JVML Opcode set}
\end{table}

\subsection{An Example}
Here we give a very simple example to exaplain how aspectj advice is woven.
\begin{multicols}{2}
\begin{lstlisting}[language=java,caption=original file Hello.java]
public class Hello {

	public static void sayHello(){
		System.out.println("hello");
	}
	
	
	public static void main(String[] args) {
		sayHello();
	}
}
\end{lstlisting}

\begin{lstlisting}[language=aspectj,caption=aspectj file World.aj]
public aspect World {
	pointcut greeting() : execution(* Hello.sayHello(..));

	after() returning() : greeting() {
		System.out.println(" world!");
	}
	
	before():greeting(){
		System.err.println("hi,");
	}
}
\end{lstlisting}
\end{multicols}

The source file \texttt{Hello.java} can only output \textit{hello} without aspectj advice.However when we compile the source file with the aspectj \texttt{World.aj} using \texttt{javac} and \texttt{ajc},the new program's output changes to \textsl{hi,hello world!}.

So how did it happen? The key is that we add a \texttt{pointcut} \textsl{greeting} in the aspectj file, which matches the \texttt{join point} \textsl{sayHello}(a static method) in file \texttt{Hello.java}. There is also a designator called \textbf{execution} that matches the exact \textit{execution} site of the code. This pointcut corresponds to a so-called cross cut concern of the original file, but notice that it is defined in the aspectj file.With the advice kind like \textbf{before()} we get an {advice},which usually will do some additions in respect to original files.In this example,it means when the stack of JVM runs into the execution of method \texttt{sayHello}(this designator is \texttt{mexecution}),there will be a additional behaviour that output \textsl{hi,} before the output of \textsl{hello}. The \texttt{after} advice is similar except that it is called just before the context of method \texttt{sayHello} is popped.

So the common practice when we using Aspectj(or generally AOP),we need firstly define \textbf{pointcuts} in the aspect file to match \texttt{join point} in the original file.Then we write \texttt{advice} for those \texttt{pointcuts}.Note that advice and pointcut is not one-to-one mapped,that is to say the relationship of them can be $1-n or n-1 or m-n$. If all works fine,we weave our advice into the original bytecode. The benefit is that we do not need to rewrite the existent source file while we change the behaviour.

Let's look the minor differences in the bytecode level. By using \texttt{javap -c} we get the disassamble infomation of \texttt{sayHello()} in \texttt{Hello} class.

\begin{lstlisting}[language=java,caption=sayHello() function in Hello class file]
  public static void sayHello();
    Code:
       0: invokestatic  #43                 // Method date20120901/AspectjHello/World.aspectOf:()Ldate20120901/AspectjHello/World;
       3: invokevirtual #49                 // Method date20120901/AspectjHello/World.ajc$before$date20120901
       _AspectjHello_World$2$f69f5afa:()V
       6: getstatic     #9                  // Field java/lang/System.out:Ljava/io/PrintStream;
       9: ldc           #15                 // String Hello
      11: invokevirtual #17                 // Method java/io/PrintStream.println:(Ljava/lang/String;)V
      14: nop
      15: invokestatic  #43                 // Method date20120901/AspectjHello/World.aspectOf:()Ldate20120901/AspectjHello/World;
      18: invokevirtual #46                 // Method date20120901/AspectjHello/World.ajc$afterReturning$date20120901
          _AspectjHello_World$1$f69f5afa:()V
      21: return
\end{lstlisting}

The index \texttt{0,3} and \texttt{15,18} describes the added bytecode to the original class file,which is what our advice \texttt{before} and \texttt{after} have done.We can also look at in the class file which \texttt{ajc} compiles.

\begin{lstlisting}[language=java,caption=aspectOf() in World class file]

    public static date20120901.AspectjHello.World aspectOf();
    Code:
       0: getstatic     #63                 // Field ajc$perSingletonInstance:Ldate20120901/AspectjHello/World;
       3: ifnonnull     19
       6: new           #65                 // class org/aspectj/lang/NoAspectBoundException
       9: dup
      10: ldc           #67                 // String date20120901_AspectjHello_World
      12: getstatic     #15                 // Field ajc$initFailureCause:Ljava/lang/Throwable;
      15: invokespecial #69                 // Method org/aspectj/lang/NoAspectBoundException."<init>":
          (Ljava/lang/Strin g;Ljava/lang/Throwable;)V
      18: athrow
      19: getstatic     #63                 // Field ajc$perSingletonInstance:Ldate20120901/AspectjHello/World;
      22: areturn
\end{lstlisting}

\begin{lstlisting}[language=java,caption=before advice in World class file]
   public void ajc$before$date20120901_AspectjHello_World$2$f69f5afa();
   Code:
      0: getstatic     #56                 // Field java/lang/System.err:Ljava/io/PrintStream;
      3: ldc           #59                 // String hi
      5: invokevirtual #48                 // Method java/io/PrintStream.println:(Ljava/lang/String;)V
      8: return
\end{lstlisting}



\subsection{Reduce Advice Kind}
In addition, we also reduce the kind of advice to \texttt{before} and \texttt{after}. In fact,we only consider the \texttt{after} advice if the execution of a join point completes normally,and that is to say we only cope with the \texttt{after returning} advice,\texttt{after throwing} and the simple \texttt{after} is neglected.The \texttt{around} advice can be modeled as the combination of \texttt{before} and \texttt{after},however it is not proven here.
The \texttt{privilege properties,aspect inheritance} and other aspectj features are also ignored here.