%compile by latex+dvipdf
\documentclass{article}
% An example how to use the calendar library and modify the layout, i.e. put
% Sunday as the first week day.
%
% Author: Berteun Damman

\usepackage{tikz}
\usetikzlibrary{calendar}
\begin{document}
    \makeatletter

    % This way you can define your own conditions, for example, you
    % could make something as `full moon', `even week', `odd week',
    % et cetera. In principle. The math in TeX could be hard.
    \pgfkeys{/pgf/calendar/start of year/.code={%
        \ifnum\pgfcalendarifdateday=1\relax%
            \ifnum\pgfcalendarifdatemonth=1\relax\pgfcalendarmatchestrue\fi%
        \fi%
    }}%

    % Define our own style
    \tikzstyle{week list sunday}=[
        % Note that we cannot extend from week list,
        % the execute before day scope is cumulative
        execute before day scope={%
               \ifdate{day of month=1}{\ifdate{equals=\pgfcalendarbeginiso}{}{
               % On first of month, except when first date in calendar.
                   \pgfmathsetlength{\pgf@y}{\tikz@lib@cal@month@yshift}%
                   \pgftransformyshift{-\pgf@y}
               }}{}%
        },
        execute at begin day scope={%
            % Because for TikZ Monday is 0 and Sunday is 6,
            % we can't directly use \pgfcalendercurrentweekday,
            % but instead we define \c@pgf@counta (basically) as:
            % (\pgfcalendercurrentweekday + 1) % 7
            \pgfmathsetlength\pgf@x{\tikz@lib@cal@xshift}%
            \ifnum\pgfcalendarcurrentweekday=6
                \c@pgf@counta=0
            \else
                \c@pgf@counta=\pgfcalendarcurrentweekday
                \advance\c@pgf@counta by 1
            \fi
            \pgf@x=\c@pgf@counta\pgf@x
            % Shift to the right position for the day.
            \pgftransformxshift{\pgf@x}
        },
        execute after day scope={
            % Week is done, shift to the next line.
            \ifdate{Saturday}{
                \pgfmathsetlength{\pgf@y}{\tikz@lib@cal@yshift}%
                \pgftransformyshift{-\pgf@y}
            }{}%
        },
        % This should be defined, glancing from the source code.
        tikz@lib@cal@width=7
    ]

    % New style for drawing the year, it is always drawn
    % for January
    \tikzstyle{year label left}=[
        execute before day scope={
            \ifdate{start of year}{
                \drawyear
            }{}
        },
        % Right align
        every year/.append style={
            anchor=east,
        }
    ]

    % Style to force giving a month a year label.
    \tikzset{draw year/.style={
        execute before day scope={
            \ifdate{day of month=1}{\drawyear}{}
        }
    }}

    % This actually draws the year.
    \newcommand{\drawyear}{
        \pgfmathsetlength{\pgf@x}{\tikz@lib@cal@xshift}%
        \pgftransformxshift{-\pgf@x}
        % \tikzyearcode is defined by default
        \tikzyearcode
        \pgfmathsetlength{\pgf@x}{\tikz@lib@cal@xshift}%
        \pgftransformxshift{\pgf@x}
    }

    \makeatother

    % The actual calendar is now rather easy:
    \begin{tikzpicture}[every calendar/.style={
            month label above centered,
            month text={\textit{\%mt}},
            year label left,
            every year/.append style={font=\Large\sffamily\bfseries,
                green!50!black},
            if={(Sunday) [blue!70]},
            week list sunday,
        }]
        \matrix[column sep=1em, row sep=1em] {
            \calendar[dates=2012-01-01 to 2012-01-last,draw year]; &
            \calendar[dates=2012-02-01 to 2012-02-last]; &
            \calendar[dates=2012-03-01 to 2012-03-last]; \\
            \calendar[dates=2012-04-01 to 2012-04-last]; &
            \calendar[dates=2012-05-01 to 2012-05-last]; &
            \calendar[dates=2012-06-01 to 2012-06-last]; \\
            \calendar[dates=2012-07-01 to 2012-07-last]; &
            \calendar[dates=2012-08-01 to 2012-08-last]; &
            \calendar[dates=2012-09-01 to 2012-09-last]; \\
            \calendar[dates=2012-10-01 to 2012-10-last]; &
            \calendar[dates=2012-11-01 to 2012-11-last]; &
            \calendar[dates=2012-12-01 to 2012-12-last]; \\
        };
    \end{tikzpicture}
\end{document}
