\section{Environments and Configurations}

\subsection{Notations}
Since the \texttt{ajc} compiler also compiles the aspectj file into java class file(in the above example,the \texttt{World.aj} is compiled into \texttt{World.class}),we consider that an aspect can be viewed as a class and its advices are represented by the methods of the class. In this case,we transfer our focus into the class files.

Firstly we give some notations that will be needed in later discussions.
\begin{itemize}
  \item $X\xrightarrow{m}{Y}$ denotes the set of all \texttt{partital functions}(also known as \texttt{maps}) from set $X$ to $Y$. For $m\in X\xrightarrow{m}{Y}$,it can be extended as $m=[x_0\mapsto y_0,\ldots x_{n-1}\mapsto y_{n-1}]$,where $x_i\in X$ and $y_i\in Y,i\in[0\ldots n-1]$.
  \item For a map $m\in X\xrightarrow{m}{Y}$,$Dom(m)=X$
  \item Given a map $f$,write $f[x\mapsto v]$ to denote the updating operation of $f$ that yields a map that is equivlent to $f$ except that $x$ is from now associated with $v$
  \item Given a record space $D=\langle f_1:D_1,\ldots,f_n:D_n\rangle$(where $f_i$ is a field with the type $D_i$ defined in the specification),and an element $e$ of type $D$,the access to the filed $f_i$ of $e$ is written as $e.f_i$ and the update of fields $f_{i1},\ldots,f_{ik}$ is written $e[f_{i1}\leftarrow v_{i1},\ldots,f_{i}\leftarrow v_{ik}]$.
  \item $(\tau)-set$ and $(\tau)-list$ denote the \texttt{set} or \texttt{list} of type $\tau$ respectively.
\end{itemize}

\subsection{Pointcuts, Joinpoint Shadows and Advices}
As with the static designators we talk about here,there are 3 cases of join point shadows.
\begin{enumerate}
  \item The case where the shadow is \textsl{exactly one} instruction
  \item The case where the shadow is an \textsl{entire} method
  \item The case that does not by themselves define shadows.They use the shadows defined by the other 6 static pointcuts.
\end{enumerate}

The \texttt{before} advice code segment is inserted after the \texttt{impdep1} instruction.\texttt{impdep1} is a reserved word in java bytecode opcode but can never appear in any class file; we assume the front-end compiler produces the instructioin for us to insert the advice;and after the compilation of the back-end compiler,the advice is woven and \texttt{impdep1} is delimitted.The \texttt{after} advice code is easy to recognize since it always usually appears with method return opcodes like \texttt{return,ireturn,areturn}.

Table 3 refers to the pointcut and the afftected area while the box below includes the syntax of our defined pointcuts.

\begin{table}
\centering
\begin{tabular}{|c|c|}
\hline Join Point Designator  & Join Point shadow\\
\hline \texttt{mcall}  & \texttt{invokevirtual} $i$,\texttt{invokespecial} $i$, \texttt{invokestatic} $i$, \texttt{invokeinterface} $i,n$\\
\hline \texttt{get} & \texttt{getfield} $i$ ,\texttt{getstatic} $i$\\
\hline \texttt{set} & \texttt{putfield} $i$ ,\texttt{putstatic} $i$\\
\hline \texttt{mexecution} & Entire method code\\
\hline \texttt{execution} & Entire advice's code\\
\hline \texttt{staticInit} & Entire ``\texttt{clinit}'' method code\\
\hline
 \end{tabular}
\caption{pointcut and affected bytecode}
\end{table}

\begin{center}
\begin{boxedminipage}{.8\columnwidth}
\begin{align*}
BasePcut \models&  \texttt{mcall}(MethodPattern) \oo \texttt{mexecution}(MethodPattern)\oo \texttt{aexecution()}\\
&\oo   \texttt{withincode}(MethodPattern)\oo \texttt{get}(FieldPattern)\oo  \texttt{set}(FieldPattern)\\
BooleanPcut  \models &  Pcut\ww \texttt{or}\ww Pcut \oo \texttt{not}\ww Pcut\oo Pcut\ww \texttt{and}\ww Pcut\\
MethodPattern  \models & \langle methodModifiers:(Methodmodifier)-\texttt{set},methodSgnature:MethodSignature,\\
& componentType:ComponentType\rangle\\
FieldPattern   \models & \langle fieldModifiers:(FieldModifier)-\texttt{set},FieldSignature:FieldSignature,\\
&componentType : ComponentType \rangle \\
MethodSignature  \models & \langle name:\texttt{Identifier},argumentsType:(Type)-\texttt{list},resultType:Type \rangle\\
FieldSignature  \models & \langle name:\texttt{Identifier},type:Type \rangle\\
MethodModifier  \models & \texttt{public} \oo \texttt{private} \oo  \texttt{static} \oo \texttt{synchronized}\\
FieldModeifer  \models & \texttt{public} \oo \texttt{private} \oo \texttt{static} \\
ComponentType  \models & ReferenceType \oo AspectType\\
ResultType  \models & Type \oo \texttt{void}\\
Type  \models & PrimitiveType \oo ReferenceType\\
ReferenceType  \models & ClassType \oo InterfaceType\\
ClassType  \models & \texttt{Identifier}\\
InterfaceType  \models & \texttt{Identifier}\\
AspectType  \models & \texttt{Identifier}\\
PrimitiveType  \models & \texttt{Identifier}
\end{align*}
\end{boxedminipage}
\end{center}

\subsection{Environments}
Below describes the \texttt{environment} before weaving, it contains the \texttt{javaEnvironment} and the \texttt{advices}.We can denote the environment as:
$$ Environment \models \langle javaEnvironment:JavaEnvironment, advices:(AdviceInfo)-\texttt{list}\rangle$$
The \texttt{javaEnvironment} is somewhat like the actual jvm environment except that it's reduced and there is an additional instruction \texttt{impdep1}.
In our model,a class is just a record containing a constant pool, a super-class, a set of interfaces, a list of fields, a map that associates values to static fields, a list of methods, three flags that indicate whether the class is initialized or not, is an interface or not, is an aspect or not and a monitor. We only explain some productions,details can been seen in \cite{web:jvm_spec}.
\begin{itemize}
  \item A \texttt{constant pool} is a map that associates a set of integers with a set of constant pool entries,which is like this:
      $$ConstantPool \models Int\xrightarrow{m}{ConstantPoolEntry}$$
      In our example,we can use \texttt{javap -verbose Hello.class} to see what contains in the constant pool.
      \begin{lstlisting}[language=java,caption=constant pool]
   #1 = Class              #2             //  date20120901/AspectjHello/Hello
   #2 = Utf8               date20120901/AspectjHello/Hello
   #3 = Class              #4             //  java/lang/Object
   #4 = Utf8               java/lang/Object
   #5 = Utf8               sayHello
   #6 = Utf8               ()V
   #7 = Utf8               org.aspectj.weaver.MethodDeclarationLineNumber
   #8 = Utf8               Code
   #9 = Fieldref           #10.#12        //  java/lang/System.out:Ljava/io/PrintStream;
  #10 = Class              #11            //  java/lang/System
  #11 = Utf8               java/lang/System
  #12 = NameAndType        #13:#14        //  out:Ljava/io/PrintStream;
  #13 = Utf8               out
  ...
      \end{lstlisting}
      The index first appearing in each line is the \textsl{key} \texttt{Integer}(\#5 for instance) and the content(like \texttt{sayHello}) can be viewed as the \textsl{value} of the entries.
  \item A \texttt{constant pool entry} can be a class type, a pair of a method signature and a supposed class.Those entries are usually defined by the \texttt{type} and the concatenation of String of entries. The recognition and interpretation, however is the task of JVM, which is out of our topic.
  \item The \texttt{monitor} associated with a class is a record of three components: \texttt{threadOwner}, \texttt{depth} and a \texttt{waitList}, i.e. it is can be represented as a tuple like below:
      $$Monitor \models  \langle threadOwner:Threadowner,depth:Nat,waitList:WaitingList\rangle$$
      The monitor is set to $\langle None,0,[\;]\rangle$ if the class is not locked.
\end{itemize}
Other productions can be seen as follows:
\begin{center}
\begin{boxedminipage}{.8\columnwidth}
\begin{align*}
JavaEnvironment \models &  ComponentType \xrightarrow{m}{Class} \\
Class \models & \langle constantPool:ConstantPool,superClass:ClassType|NoneType,\\
&interfaces:(ClassType)-\texttt{set},fields:(Field)-\texttt{set},\\
&staticMap:Field\xrightarrow{m}{Value},methods:(Method)-\texttt{set},\\
&initialized:Int,interface:Int,aspect:Int,monitorClass:Monitor\rangle\\
ConstantpoolEntry \models & ClassType\oo MethodPoolEntry,FieldPoolEntry\\
MethodPoolEntry \models & \langle methodSignature:MethodSignature,supposedClass:ClassType\rangle\\
FieldPoolEntry \models & \langle fieldSignature:FieldSignature,supposedClass:ClassType\rangle\\
ThreadOwner \models& ThreadId|NoneType\\
WaitingList \models &(ThradId)-\texttt{list}\\
ThreadId \models & Nat\\
Field \models & \langle fieldSignagure:FieldSignature,fromClass:ComponentType,\\
& fieldModifiers:(FieldModifier)-\texttt{set}\rangle\\
Method \models & \langle methodSignagure:MethodSignature,fromClass:ComponentType,\\
&methodModifiers:(MethodModifier)-\texttt{set},code:Code,method\\
&Variables:MethodVariables\rangle\\
AdviceInfo \models & \langle kind:\left\{ \texttt{Before,After} \right\},pointcut:Pcut,\\
&fromClass:AspectType,adviceSignagure:MethodSignagure\rangle\\
Code \models & ProgrammCounter\xrightarrow{m}{Instruction}\\
Instruction \models & JVMLInstruction|\texttt{impdep1}\\
ProgramCounter \models & Nat\\
MethodVariables \models & (Value)-list\\
\end{align*}
\end{boxedminipage}
\end{center}


\subsection{Configurations}
The operational semantics is based on the evolution of configurations that are defined hereafter. Weaving a class with some aspects is the result of weaving all its methods with the considered aspects. So we only need to describe the weaving inside one method and a configuration will have the following form:
$$\langle\xi,m,pc,ads,nextpc\rangle$$
where:
\begin{itemize}
  \item $\mathbf{\xi}$ represents the environment
  \item $\mathbf{m}$ is the related method
  \item $\mathbf{pc}$ represents the program counter that contains the address of the instruction to be advised into $m$
  \item $\mathbf{ads}$ represents the advices to consider
  \item $\mathbf{nextpc}$ represents the program counter for the next instruction to consider
\end{itemize}